\documentclass[a4paper]{article}

\usepackage[utf8]{inputenc}
\usepackage{graphicx}
\usepackage[frenchb]{babel}
\usepackage{amsmath}

\begin{document}

\title{VISA -- TP1 : éléments de géométrie projective et calibration de caméra}
\author{Arnaud Cojez}
\date{}

\maketitle

\newpage
\tableofcontents
\newpage
%----------------------------------------------------------------------------------------
%	INTRODUCTION
%----------------------------------------------------------------------------------------

\section{Introduction}

\subsection{Motivation}
Dans le domaine de la vision artificielle, calibrer le périphérique d'acquisition (ici une caméra) est une étape indispensable, nécessaire pour l'extraction de données provenant d'une image en 2 dimensions.
Il s'agit de déterminer quelle transformation permet de passer d'un élément 3D à sa projection en 2D.

La transformation en question dépend de plusieurs paramètres :
\begin{itemize}
  \item Les propriétés de l'élément, ainsi que sa situation dans l'espace ;
  \item Les coordonnées intrinsèques, propres au périphérique d'acquisition ;
  \item Les coordonnées extrinsèques, donnant la situation du périphérique dans l'espace.
\end{itemize}

Il arrive que nous ne soyons pas en possession des coordonnées intrinsèques, ni extrinsèques du périphérique d'acquisition. Ces données sont communiquées ou non par le fabricant.

Afin de retrouver ces paramètres, plusieurs méthodes sont disponibles. Nous allons nous intéresser ici à la méthode de Zhang.

Nous disposons de fichiers texte listant les propriétés d'une mire dans l'espace 3D. 4 images ont été créés à partir de cette mire, nous disposons également de leurs propriétés.

\subsection{La Méthode de Zhang}

Cette méthode retient notre attention pour plusieurs raisons. En effet, celle-ci est simple à mettre en oeuvre et peu couteuse :
La technique nécessite uniquement la caméra et un motif planaire (dont on connait les propriétés) représenté dans au moins 2 orientations différentes. Au contraire des méthodes classiques qui nécessitaient par exemple des mires en 3 dimensions.
Elle est également flexible :
Le motif peut être facilement imprimé puis attaché à n'importe quelle surface planaire. De plus, la caméra et le plan peuvent être bougés librement.

Cette méthode est découpée en 3 grandes étapes :
Pour chaque image :
\begin{itemize}
  \item On va calculer l'homographie (transformation entre 2 plans ayant les mêmes propriétés, ici des plans 2D) entre la scène et la projection ;
  \item À partir de l'homographie et des informations qui en seront extraites, on pourra calculer la matrice de paramètres intrinsèques ;
  \item On déterminera finalement la matrice de paramètres extrinsèques propre à chaque image.
\end{itemize}

Ces étapes seront détaillées dans le présent document.

\clearpage
%----------------------------------------------------------------------------------------
%	DÉTERMINATION DES CONTRAINTES À PARTIR DE L'HOMOGRAPHIE
%----------------------------------------------------------------------------------------

\section{Homographie et contraintes}

Une homographie est une transformation linéaire entre deux plans projectifs. C'est à dire qu'il n'y aura pas de perte ni d'ajout d'informations. Celles existantes sont juste transformées pour être comprises dans un nouveau plan, aux mêmes propriétés que le premier. Les lignes restent des lignes, les points restent des points, etc. Cependant, les angles et les distances sont modifiés

Dans notre cas, cela correspond à la transformation entre le motif imprimé et la projection captée par la caméra.

Pour chaque image, nous calculerons la matrice d'homographie.

\begin{equation}
  \begin{bmatrix}
    h_1 & h_2 & h_3
  \end{bmatrix}
  = \lambda A
  \begin{bmatrix}
    r_1 & r_2 & t
  \end{bmatrix}
\end{equation}

\begin{equation}
  \begin{aligned}
    h_1^T A^{-T} A^{-1} h_2 &= 0 \\
    h_1^T A^{-T} A^{-1} h_1 &= h_2^T A^{-T} A^{-1} h_2
  \end{aligned}
\end{equation}

À partir de cette matrice, nous pouvons obtenir 2 contraintes d'homographie, qui nous serviront à calculer la matrice intrinsèque.

\begin{equation}
  v_{ij} = [h_{i1} h_{j1} , h_{i1} h_{j2} + h_{i2} h_{j1} , h_{i2} h_{j2} ,
h_{i3} h_{j1} + h_{i1} h_{j3} , h_{i3} h_{j2} + h_{i2} h_{j3} , h_{i3} h_{j3} ]^T
\end{equation}

Nous ajoutons enfin ces contraintes à une matrice V. Puis nous recommençons l'opération avec une image différente.
- Montrer l'implémentation des contraintes

\clearpage
%----------------------------------------------------------------------------------------
%	DÉTERMINATION DE LA MATRICE INTRINSÈQUE
%----------------------------------------------------------------------------------------

\section{Matrice Intrinsèque}

La matrice intrinsèque représente les paramètres propres à la caméra. Ces paramètres ne sont pas forcément connus mais peuvent être retrouvés grâce à la méthode de Zhang.

\begin{equation}
  A=
  \begin{bmatrix}
   \alpha & \gamma & u_0 \\
   0 & \beta & v_0 \\
   0 & 0 & 1 \\
  \end{bmatrix}
\end{equation}

\begin{itemize}
\item $\alpha$ et $\beta$ sont les facteurs d'agrandissement de l'image multipliés par la distance focale ($k_u$ * f, $k_v$ * f). C'est à dire la taille que représente un pixel sur l'image réelle (par exemple, 5mm par pixel) ;
\item $u_0$ et $v_0$ représentent l'origine de l'image projetée sur les capteurs. Il faut se déplacer de ($u_0$, $v_0$) à partir du centre afin de retrouver l'origine de l'image ;
\item $\gamma$ traduit l'angle dans lequel sont placés les capteurs photosensibles de la caméra les uns par rapport aux autres, multiplié par la distance focale ($s_{uv}$ * f). La plupart du temps, l'angle est un angle droit, ce qui fait que ce paramètre est négligé et prend donc une valeur nulle.
\end{itemize}

Ces paramètres sont calculés à partir des équations suivantes, en se servant d'une matrice B, matrice calculée à l'aide des termes de contraintes trouvés dans la partie précédente :
\begin{equation}
  \begin{aligned}
    v_0 &= (B_{12}B_{13} - B_{11}B_{23}) / (B_{11}B_{22} -B_{12}^2) \\
    \delta &= B_{33} - [B_{13}^2 + v_0(B_{12}B_{13} - B_{11}B_{23})] / B_{11} \\
    \alpha &= \sqrt{\lambda/B_{11}} \\
    \beta &= \sqrt{\lambda B_{11}/(B_{11}B{22}/B{12}^2}) \\
    \gamma &= -B_{12}\alpha^2\beta/\lambda \\
    u_0 &= \gamma v_0 /\beta - B_{13}\alpha^2\lambda
  \end{aligned}
\end{equation}

- Montrer l'implémentation matrice intrinsèque

\clearpage
%----------------------------------------------------------------------------------------
%	DÉTERMINATION DE LA MATRICE EXTRINSÈQUE
%----------------------------------------------------------------------------------------

\section{Matrice Extrinsèque}
Nous pouvons désormais récupérer les paramètres extrinsèques de la caméra.
Ces paramètres varient en fonction de la position et de la rotation du périphérique dans l'environnement.

\begin{equation}
  \begin{aligned}
    r_1 &= \lambda A^{-1} h_1 \\
    r_2 &= \lambda A^{-1} h_2 \\
    r_3 &= r_1 * r_2 \\
    t &= \lambda A^{-1} h_3
  \end{aligned}
\end{equation}

\begin{itemize}
\item
\begin{equation}
  \begin{bmatrix}
    r_1, r_2, r_3
  \end{bmatrix}
\end{equation}
est la matrice de rotation permettant de passer du repère lié à l'espace de travail au repère lié à la caméra ;

\item
\begin{equation}
  \begin{bmatrix}
    t_x, t_y, t_z
  \end{bmatrix}
\end{equation}
 représente la translation permettant de passer du repère lié à la scène au repère lié à la caméra.
\end{itemize}

- Montrer l'implémentation

\clearpage
%----------------------------------------------------------------------------------------
%	RÉSULTATS OBTENUS
%----------------------------------------------------------------------------------------

\section{Résultats}

\subsection{Matrice Intrinsèque}

Résultat obtenu :
  $\begin{bmatrix}
      3498.2767 & - 3.1310503  &  336.76583 \\
      0.        &   3503.8946  &  220.1142 \\
      0.        &   0.         &  1.
  \end{bmatrix}$

Résultat attendu :
  $\begin{bmatrix}
    3546.099291 & 0.000000 & 320.000000 \\
    0.000000 & 3546.099291 & 240.000000 \\
    0.000000 & 0.000000 & 1.000000
  \end{bmatrix}$

On constate que le résultat obtenu est très proche du résultat attendu.

\subsection{Matrices Extrinsèques}

Image 1 :
Résultat obtenu :
$\begin{bmatrix}
  0.9999998 & 0.0009052 & 0.0009052 & - 48.811566 \\
  0.0000377 & 0.9982948 & 0.0000376 & 54.733308 \\
  0.0006696 & - 0.0015763 & - 0.0000011 & 9854.3605
\end{bmatrix}$
Résultat attendu :
Translation = (0, 0, 10000), Rotation = (0, 0, 0)

Image 2 :
$\begin{bmatrix}
  0.7124496 & 0.0007762 & 0.0005530 & - 46.123208 \\
  - 0.0039703 & 1.0010299 & - 0.0039744 & 43.83032 \\
  - 0.7017120 & - 0.0006379 & 0.0004476 & 7905.8899
\end{bmatrix}$
Résultat attendu :
Translation = (0, 0, 8000), Rotation = (0, 0.785398, 0)

Image 3 :
$\begin{bmatrix}
  0.9848432 & 0.1745697 & 0.1719238 & - 43.812298 \\
  - 0.1734468 & 0.9833136 & - 0.1705526 & 49.196566 \\
  0.0002832 & 0.0005524 & 0.0000002 & 8870.6728
\end{bmatrix}$
Résultat attendu :
Translation = (0, 0, 9000), Rotation = (0, 0, -0.174532)

Image 4 :
$\begin{bmatrix}
  1. & 0.0045336 & 0.0045336 & - 143.86961 \\
  0.0000023 & 0.7020868 & 0.0000016 & 42.078156 \\
  - 0.0000609 & - 0.714386 & 0.0000435 & 8872.4252
\end{bmatrix}$
Résultat attendu :
Translation = (-100, 0, 9000), Rotation = (-0.785398, 0, 0)

On constate ici aussi que les translations calculées (colonne de droite de chaque matrice) sont équivalentes aux translations attendues.
De même pour les rotations (3 colonnes de gauche), celles-ci approchent de 0 dans les deux cas.

\clearpage
%----------------------------------------------------------------------------------------
%	CALIBRATION FOCALE
%----------------------------------------------------------------------------------------

\section{Zhang simplifié}

Question :
Si on suppose que ces paramètres sont connus par construction (taille du capteur, nombre de pixels, etc.), indiquer comment on peut modifier la méthode de Zhang afin de ne déterminer que la distance focale de la caméra (exprimée en unités pixel).

On sait que la matrice intrinsèque
\begin{equation}
  A=
  \begin{bmatrix}
   \alpha & \gamma & u_0 \\
   0 & \beta & v_0 \\
   0 & 0 & 1 \\
  \end{bmatrix}
  =
  \begin{bmatrix}
   k_uf & s_{uv}f & u_0 \\
   0 & k_vf & v_0 \\
   0 & 0 & 1 \\
  \end{bmatrix}
  \begin{bmatrix}
   f & 0 & 0 \\
   0 & f & 0 \\
   0 & 0 & 1 \\
  \end{bmatrix}
\end{equation}

On sait également que les facteurs d'échelles $k_u$ et $k_v$ correspondent (respectivement en x et en y) à :
\begin{equation}\frac{nombre\ de\ pixels}{taille\ du\ pixel}\end{equation}

Comme on connait ces valeurs, on peut donc retrouver la valeur de la focale en trouvant $\alpha$, $\beta$ et en résolvant une des opérations suivantes :
\begin{equation}f=\frac{\alpha}{k_u}=\frac{\beta}{k_v}\end{equation}


\clearpage
%----------------------------------------------------------------------------------------

%----------------------------------------------------------------------------------------
%	CONCLUSION
%----------------------------------------------------------------------------------------

\section{Conclusion}

La méthode de Zhang nous permet donc de trouver les paramètres propres à la caméra, mais aussi de situer la caméra par rapport à la mire dans la scène.
Nous avons constaté qu'en plus d'être peu couteuse et simple d'utilisation, cette méthode apportait des résultats s'approchant des propriétés attendues.
Ces raisons nous font comprendre que la méthode de Zhang est une méthode particulièrement recommandée afin de calibrer une caméra.

\clearpage
%----------------------------------------------------------------------------------------

\end{document}
